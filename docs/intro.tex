\section{Introdução}

Este documento é sobre a ferramenta \ipmt. Essa ferramenta é capaz de
pré-processar um arquivo de texto, gerando um índice. Sucessivas buscas podem
ser feitas através desse índice, sem a necessidade de percorrer o texto
novamente.


\ipmt primeiro gera um índice para o texto usando o algoritmo LSA (Linear Suffix
Array), depois esse índice é comprimido em um arquivo juntamente com o texto
usando o algorítmo \lz. Para realizar buscas, primeiramente o arquivo é
descomprimido usando o lz78-decode e depois o casamento de padrões é realizada
de acordo com o LSA. 


Os integrantes da equipe foram responsáveis pelas seguintes tarefas:
\begin{itemize}
\item João: 

\item Paulo: Implementação do algorítmo \lz e da interface de comunicação entre os algoritmos.

\item Raul: Implementação da estrutura de indexação LSA.

\end{itemize}

