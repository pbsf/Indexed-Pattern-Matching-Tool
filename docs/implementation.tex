\section{Implementação}
Todos os algoritmos foram implemetados em C++ por questões de eficiência. A
implementação em Python feita em sala de aula foi usada como uma base inicial.

\subsection{Descrição dos algoritmos implementados}

\subsubsection{\lsa}
O algoritmo de indexação  implementado teve como
base~\cite{KarkkainenS03} para a construção em
tempo linear de um array de sufixos. Em suma, o array de sufixos é um array de
inteiros que armazena a permutação de n índices ordenados lexicograficamente,
onde n é o tamanho do texto. Uma vez construído o array de sufixos, a
complexidade da busca passa a ser linear com relação ao tamanho do padrão.

\subsubsection{\lst}
O algoritmo de indexação implementado teve como
base ~\cite{UkkonenST} para a construção em tempo linear
de uma árvore de sufixos. A estrutura implementada representa
todos os sufixos de uma cadeia. A implementação contém alguns truques
para que a construção seja feita em tempo linear. Um desses truque é adicionar aos
nós suffix links, também chamados como transições de falha ou fronteiras. Devido 
ao alto consumo de memória ao gerar a árvore de sufixo, resolvemos deixar a feature
de melhorar o gerenciamento de memória para o futuro. Segundo ~\cite{BogdanCraig} se o núcleo 
da implementação for orientada a objeto, a árvore de sufixo apresenta efeitos indesejáveis
de memória fragmentada.

\subsubsection{\lz}

O algoritmo de compressão \lz teve como base a implementação vista em sala de
aula e descrita em~\cite{Storer:1987:DCM:42791}. O \lz utiliza um dicionário
dinâmico explícito, onde a referência compreende um par composto pelo índice no
dicionário e o caracter de mismatch.

Durante a compressão (\lz-encode), o dicionário é criado dinâmicamente a cada
mismatch. Junto com o dicionário, também é criado um código que representa a
string que está sendo comprimida. O \lz-encode é linear de acordo
com o tamanho da string que está sendo comprimida.

O processo de descompressão (\lz-decode) recebe somente o código gerado durante
a compressão, e é capaz de gerar o dicionário dinamicamente, bem como a string
original que foi comprimida. O \lz-decode é linear de acordo com o tamanho do
código recebido na entrada.


\subsection{Detalhes de implementação}
Abaixo descrevemos algumas decisões e peculiaridades de cada algoritmo.

\subsubsection{\lsa}

Na construção do \lsa há uma etapa de criação de dois arrays de sufixos, S1 e S2.
Seja index a posição de um caracter em um texto:  A função buildS1andS2 constrói
o array de sufixo S1 que contém sufixos tal que index \% 3 = 0 e também constrói
o array de sufixo S2 que contém sufixos tal que index \% 3 != 0. Após a
construção de S1 e S2, estes são ordenados através de uma implementação do Radix
Sort com o objetivo de otimizar essa etapa. Como o Radix Sort não faz
comparações entre valores, nesse contexto, o seu desempenho é superior a um
algoritmo de ordenação por comparação. A ordenação de S1 e S2 foi necessária
para a etapa de merge (S1  U  S2 = SA)  de tal forma que o custo do merge é
realizado em tempo linear. Após o merge, obtemos os índices devidamente
ordenados.


\subsubsection{\lz}
O dicionário dinâmico possui uma estrutura de Trie: Cada nó mapeia um índice a
somente um char, e possui nós descendentes de forma que o nó original e cada um
de seus descendes forma uma sequência diferente de caracteres encontrada no
texto.

Usamos como alfabeto do código de saída o sistema binário. Como cada elemento do
código de saída tem somente um bit, usamos como estrutura de dados para guardar
o código um vector de
bool\footnote{\url{http://en.cppreference.com/w/cpp/container/vector_bool}}.
Optamos por essa estrutura de dados pois ela possui uma otimização de espaço:
Um bool em C++ ocupa 8 bits (ou 1 byte) de espaço na memória, porem um vetor de
bool usa somente 1 bit para cada elemento.

Outra peculiaridade desse algorítmo é que pode ocorrer do arquivo descomprimido
conter algum "lixo" no último byte. Isso acontece porque a escrita em arquivo só
pode ser feita de byte em byte, porém o código gerado pelo \lz-encode possui uma
sequência de bits. Caso o número de bits não seja um múltiplo de 8, o último
byte precisa ser preenchido com uma sequência de 0s, o que pode alterar o último
byte na hora da descompressão. Isso poderia ser contornado com alguma flag no
início do arquivo comprimido, informando quantos bits devem ser descartados do
último byte. Deixamos isso como trabalho futuro.

\subsection{Descrição do formato .idx}

A ferramenta \ipmt gera e lê arquivos no formato .idx. Esse arquivo é gerado da
seguinte forma para um arquivo de entrada file.txt:

\begin{enumerate}

\item Primeiramente é gerado o \lsa a partir do conteúdo de file.txt.
\item Após isso, conta-se o número de linhas de file.txt.
\item É criado um novo arquivo com o seguinte contéudo:
\begin{verbatim}
            [Número de linhas contidas em file.txt]
            [Conteúdo de file.txt]
            [Elementos do LSA separados por um espaço]

(Note que quebras de linha separam os elementos acima.)
\end{verbatim}
\item Esse novo arquivo é então comprimido usando o \lz, gerando o arquivo file.idx.

\end{enumerate}

Na hora de ler o arquivo .idx, primeiro é realizada a descompressão. Através do
resultado, a ferramenta sabe que a primeira linha contém o número de linhas do
texto. Logo, as linhas seguintes são referentes ao LSA, que é utilizado pela
busca.
