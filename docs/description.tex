\section{Descrição de uso da ferramenta \ipmt}

O projeto contém um arquivo Makefile. Após a execução do comando make, o
executável \ipmt será gerado no diretório bin. A ferramenta \ipmt possui 4 modos
de execução:

\begin{itemize}

\item Modo de indexação - \ipmt index file.txt

    O comando acima irá criar o arquivo file.idx, um arquivo comprimido que
    contém o conteúdo de file.txt e um índice para esse arquivo que possibilita
    a realização de buscas.

\item Modo de busca - \ipmt search -c herself file.idx

    O comando acima irá listar a quantidade de ocorrência do padrão "herself"
    encontradas no arquivo indexado file.idx. O argumento -c é opcional, caso
    não informado todas as linhas contendo ocorrências serão impressas.

\item Modo de compressão - \ipmt compress file.txt

    O comando acima irá comprimir o arquivo file.txt em um arquivo file.comp.

\item Modo de descompressão - \ipmt decompress file.comp

    O comando acima irá descomprimir o arquivo file.comp em um arquivo
    file.comp.decomp.

\end{itemize}

Os dois últimos modos não foram pedidos na especificação do projeto, mas nós os
criamos para facilitar a comparação do nosso algoritmo de compressão e
descompressão com algoritmos existentes.
