\section{Conclusão}

Nesse relatório foi apresentada a ferramenta \ipmt, capaz de indexar e comprimir
um arquivo, possiblitando buscas cujo tempo varia linearmente com o tamanho do
padrão.

Como esperado, conseguimos obter resultados melhores do que a ferramenta
grep, porém ainda existem pontos onde a ferramenta pode melhorar para ficar tão
eficiente quanto o codesearch.

Por exemplo, no método de contar os matches
poderia ser implementado uma busca binária em dois estágios, um que busca a
primeira ocorrência e outra que busca a última ocorrência. Logo o número de
ocorrências seria a substração entre a última e a primeira ocorrência.
Além disso, atualmente o array do LSA é transformado em uma sequência de
caracteres separados por um espaço. Ou seja, o array com objetos [1, 5, 10] será
transformado na string "1 5 10". Note que cada algarismo está utilizando um byte
para ser representado, porém sabemos que cada algarismo vai variar entre 0 e 9,
então na verdade precisamos somente de 4 bits para representar cada um,
possivelmente melhorando a taxa de compressão após a indexação.

Deixamos estas e outras melhoras como trabalho futuro.
